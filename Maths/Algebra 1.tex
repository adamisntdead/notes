\documentclass[oneside,english,course]{lecture}

\usepackage{polynom}

\title{Algebra 1}
\subtitle{Leaving Cert Higher Level Maths} 
\shorttitle{LC HL Maths - Algebra 1}
\subject{Maths}
\author{Adam Kelly}
\email{adamkelly2201@gmail.com}
\date{28}{08}{2018}
\dateend{28}{08}{2018}
\attn{These notes are not endorsed by any teachers, and I have modified them and added my own
details to these after the classes.
All errors are almost surely mine.}
\morelink{github.com/adamisntdead/notes} 

\begin{document}

\section{Polynomial Expressions}

\lecture{28}{08}{2018}

\begin{definition}[Polynomial]
	A \textbf{polynomial} is an expression of the form:

	$$
		a_x x^n + a_{n-1} x^{n - 1} + \dots + a^2 x^2 + a_1 x + a_0
	$$

	Where $a_i \in \mathbb{R}$; $i = 0, 1, 2, \dots, n$, $n \in \mathbf{N}$.
\end{definition}

There are a few types of polynomial which are useful to know:

\begin{enumerate}[a)]
	\item A \textbf{linear} polynomial is of degree 1 ($ax + b$)
	\item A \textbf{quadratic} polynomial is of degree 2 ($ax^2 + bx + c$)
	\item A \textbf{cubic} polynomial is of degree 3 ($ax^3 + bx^2 + cx + d$)
\end{enumerate}

\begin{definition}[Degree]
	The \textbf{degree} of a polynomial is the highest power of its monomials with non-zero coefficients.
\end{definition}

\subsection{Addition And Subtraction Of Polynomial Expressions}

Adding and subtracting polynomials involves the combination of like terms.

\begin{definition}[Like terms]
	Monomials that contain the same variable raised to the same power.
\end{definition}

\begin{theorem}[Difference of two squares]
	An expression of the form $a^2 - b^2$ can be factored into $(a + b)(a - b)$
\end{theorem}

\begin{proof}
	\begin{align*}
		  & (a + b)(a - b)      \\
		= & a(a - b) + b(a - b) \\
		= & a^2 - ab + ab -b^2  \\
		= & a^2 - b^2
	\end{align*}
\end{proof}

\end{document}