\documentclass[english,course]{lecture}

\usepackage{polynom}
\usepackage{cancel}

\title{Algebra 1}
\subtitle{Leaving Cert Higher Level Maths} 
\shorttitle{LC HL Maths - Algebra 1}
\subject{Maths}
\author{Adam Kelly}
\email{adamkelly2201@gmail.com}
\date{28}{08}{2018}
\dateend{2}{09}{2018}
\attn{These notes are not endorsed by any teachers, and I have modified them and added my own
details to these after the classes.
All errors are almost surely mine.}
\morelink{github.com/adamisntdead/notes} 

\begin{document}

\section{Polynomial Expressions}

\lecture{28}{08}{2018}

\begin{definition}[Polynomial]
	A \textbf{polynomial} is an expression of the form:

	$$
		a_x x^n + a_{n-1} x^{n - 1} + \dots + a^2 x^2 + a_1 x + a_0
	$$

	Where $a_i \in \mathbb{R}$; $i = 0, 1, 2, \dots, n$, $n \in \mathbf{N}$.
\end{definition}

\begin{definition}[Degree]
	The \textbf{degree} of a polynomial is the highest power of its monomials with non-zero coefficients.
\end{definition}

\\
There are a few types of polynomial which are useful to know:

\begin{enumerate}[a)]
	\item A \textbf{linear} polynomial is of degree 1 ($ax + b$)
	\item A \textbf{quadratic} polynomial is of degree 2 ($ax^2 + bx + c$)
	\item A \textbf{cubic} polynomial is of degree 3 ($ax^3 + bx^2 + cx + d$)
\end{enumerate}

\begin{definition}[Coefficent]
	A quantity that multiplies the variable in an algebraic expression (for example $4$ in $4x^3$).
\end{definition}

\begin{definition}[Constant Term]
	A term in an algebraic expression that does not contain any modifiable variables.
\end{definition}

\begin{definition}[Binomial]
	An polynomial containing two terms.
\end{definition}

\begin{definition}[Trinomial]
	An polynomial containing three terms.
\end{definition}

\subsection{Addition And Subtraction Of Polynomial Expressions}

Adding and subtracting polynomials involves the combination of like terms.

\begin{definition}[Like terms]
	Monomials that contain the same variable raised to the same power.
\end{definition}

\begin{example}[Simplify]
	$(6x^2 – 7x + 4) + (7x^2 – 9x + 8)$


	\begin{align*}
		  & (6x^2 – 7x + 4) + (7x^2 – 9x + 8)      \\
		  & = (6x^2 + 7x^2) + (-7x - 9x) + (4 + 8) \\
		  & = 13x^2 - 16x^2 + 12
	\end{align*}
\end{example}

\begin{example}[Subtract]
	$7x^3 – 6x^5 + 2x^2$ from $9x^5 – 6x^3 + 7x^2$


	\begin{align*}
		  & (9x^5 – 6x^3 + 7x^2) - (7x^3 – 6x^5 + 2x^2)         \\
		  & = (9x^5 - (-6x^5)) + (-6x^3 - 7x^3) + (7x^2 - 2x^2) \\
		  & = 15x^5 - 13x^3 + 5x^2
	\end{align*}
\end{example}

\subsection{Multiplying Polynomial Expressions}

Polynomials are multiplied using the distributive property of addition:

$$
	a(b + c) = ab + ac
$$

\begin{example}[Simplify]
	$(x + 4)(2x + 5)$


	\begin{align*}
		  & (x + 4)(2x + 5)       \\
		  & = 2x^2 + 8x + 5x + 20 \\
		  & = 2x^2 + 13x + 20
	\end{align*}
\end{example}

\subsection{Dividing Polynomial Expressions}

\lecture{30}{08}{2018}

There are a few cases for the division of polynomials.
Some of these are detailed below\\

Remember, the following equalities \textbf{do not hold} when the quotient's denominator is zero.
For example, if you have the expression $(2x + 2) \div (x + 1)$, and you simplify it to $2$, you must remember that this expression is \textbf{not equal} to two if $x = -1$, as that would be division by zero and thus undefined.

\subsubsection{Denominator Is A Factor Of Each Term}

If the denominator is a factor of each term of the numerator,
the quotient can be simplified by dividing each term of the numerator by the denominator.

\begin{example}[Simplify]
	$(-16x^4 + 8x^2 + 2x) \div 2x$


	\begin{align*}
		  & \frac{-16x^4 + 8x^2 + 2x}{2x}                         \\
		  & = \frac{-16x^4}{2x} + \frac{8x^2}{2x} + \frac{2x}{2x} \\
		  & = -7x^3 + 4x + 1
	\end{align*}
\end{example}

\subsubsection{Denominator Is A Factor Of The Numerator}

If the denominator is a factor of the numerator, the quotient can be simplified
by factoring the numerator and then canceling out the factor.
\margintext{For more on factoring, see section \ref{factoring}}

\begin{example}[Simplify]
	$(15x^2 + 22x + 8) \div (5x + 4)$


	\begin{align*}
		  & \frac{15x^2 + 22x + 8}{5x + 4}                      \\
		  & = \frac{(3x + 2)\cancel{(5x + 4)}}{\cancel{5x + 4}} \\
		  & = 3x + 2
	\end{align*}
\end{example}

\subsubsection{Polynomial Long Division}


If the other two methods don't work, you can use polynomial long division.
This is a generalized version of the arithmetic technique called long division.

\begin{example}[Simplify]\label{long-div-example}
	$(x^3 + x^2 - 2x) \div (x - 1)$


	\begin{center}
		\polylongdiv{x^3 + x^2 - 2x}{x - 1}
	\end{center}

	$$
		\therefore \frac{x^3 + x^2 - 2x}{x - 1} = x^2 + 2x
	$$
\end{example}
\\
You should note that it is possible that it doesn't divide in evenly,
in which case the division will leave a \emph{remainder}.
This can just be added at the end. Here is an example of a remainder

\begin{example}[Simplify]
	$(x^3 + x^2 - 1) \div (x - 1)$


	\begin{center}
		\polylongdiv{x^3+x^2-1}{x-1}
	\end{center}

	$$
		\therefore \frac{x^3 + x^2 - 1}{x - 1} = x^2 + 2x + 2 + \frac{1}{x - 1}
	$$

\end{example}

\subsubsection{Synthetic Division}

If the denominator of the expression is $x - a$, where $a \in \mathbb{R}$, then you can use a
division technique called synthetic division.

This technique is generally used in the process of finding the roots of a polynomial.

\begin{example}[Complete]
	example \ref{long-div-example} using synthetic division.

	\begin{center}
		\polyhornerscheme[x=1]{x^3 + x^2 - 2x}
	\end{center}

	$$
		\therefore \frac{x^3 + x^2 - 2x}{x - 1} = x^2 + 2x
	$$

\end{example}


\section{Factorising Polynomial Expressions}\label{factoring}

\lecture{31}{07}{18}

\begin{definition}[Factor]
	An algebraic factor is an expression that divides a polynomial leaving no remainder.
\end{definition}
\\

Factorising is a very important technique in algebra.
A few factoring techniques are shown below.

\subsection{Highest Common Factor}

This method can be done by looking for the highest common factor of each of the expressions and then working backwards from there.

\begin{example}[Factor]
	\begin{align*}
		2x^2 + 6x          & = 2x(x + 3)       \\
		15axy + 2xyb + 4xy & = 2xy(7a + b + 2)
	\end{align*}
\end{example}

\subsection{Grouping}

An expression can sometimes be factored by breaking the expression up into groups and factoring the individual parts.

\begin{example}[Factor]
	\begin{align*}
		6x^2y + 3xy^2 -12x - 6y & = 3xy(2x + y) - 6(2x + y) \\
		                        & = (2x + y)(2xy - 5)
	\end{align*}
\end{example}

\subsection{Difference Of Two Squares}

\begin{theorem}[Difference of two squares]
	An expression of the form $a^2 - b^2$ can be factored into $(a + b)(a - b)$
\end{theorem}

\begin{proof}
	\begin{align*}
		(a + b)(a - b)
		  & = a(a - b) + b(a - b) \\
		  & = a^2 - ab + ab -b^2  \\
		  & = a^2 - b^2
	\end{align*}
\end{proof}

\begin{example}[Factor]
	\begin{align*}
		x^2 - 36   & = (x + 6)(x - 6)                                    \\
		4x^2 - 81  & = (2x + 9)(2x - 9)                                  \\
		9x^3 - 81x & = 9x(x^2 - 9) = 9x(x + 3)(x - 3)                    \\
		16x^4 - 1  & = (4x^2 + 1)(4x^2 - 1) = (4x^2 + 1)(2x + 1)(2x - 1) \\
	\end{align*}
\end{example}

\subsection{Difference Of Two Cubes}

\begin{theorem}[Difference of two cubes]
	An expression of the form $a^3 - b^3$ can be factored into $(a - b)(a^2 + ab + b^2)$
\end{theorem}

\begin{proof}
	\begin{align*}
		(a-b)(a^2 + ab +b^2 )
		  & = a(a^2 + ab +b^2 ) - b(a^2 + ab +b^2 )                        \\
		  & = (a^3 + a^2 b + ab^2) - (a^2b + ab^2 + b^3)                   \\
		  & = (a^3 + \cancel{a^2 b + ab^2}) - (\cancel{a^2b + ab^2} + b^3) \\
		  & = a^3 - b^3
	\end{align*}
\end{proof}

\begin{example}[Factor]
	\begin{align*}
		x^3 - 8        & = (x - 2)(x^2 + 2x + 4)          \\
		125x^3 - 27y^3 & = (5x - 3y)(25x^2 + 15xy + 9y^2)
	\end{align*}
\end{example}

\subsection{Sum Of Two Cubes}

\begin{theorem}[Sum of two cubes]
	An expression of the form $a^3 + b^3$ can be factored into $(a + b)(a^2 - ab + b^2)$
\end{theorem}

\begin{proof}
	\begin{align*}
		(a+b)(a^2 - ab +b^2 )
		  & = a(a^2 - ab +b^2 ) + b(a^2 - ab +b^2 )          \\
		  & = a^3 - a^2b + ab^2 + ba^2 - ab^2 + b^3          \\
		  & = a^3 - \cancel{a^2b + ab^2 + ba^2 - ab^2} + b^3 \\
		  & = a^3 + b^3
	\end{align*}
\end{proof}

\begin{example}[Factor]
	\begin{align*}
		27x^3 + 1      & = (3x + 1)(9x^2 - 3x + 1)         \\
		81x^3 + 192b^3 & = (5x + 6b)(25x^2 - 30xb + 36b^2)
	\end{align*}
\end{example}

\subsection{Factoring With The Quadratic Formula}

If the previous methods don't work and you with to factor a quadratic, you can use ths following method.
The best way to show this is with an example.

\begin{example}[Factor]
	$3x^2 - 17x + 20$\\
	\textbf{Answer:}

	First, you must let the expression equal $0$, and then solve for $x$:

	\begin{align*}
		  & 3x^2 - 17x + 20 = 0                        \\
		  & \implies x
		= \frac{17 \pm \sqrt{(-17)^2 - 4(3)(20)}}{2(3)}
		= \frac{17 \pm 7}{6}\\
		  & \implies x = 4 \text{ or } x = \frac{5}{3}
	\end{align*}

	From this you can find that if $x = 4$, then $x- 4$ is a factor.
	If $x = \frac{5}{3} \implies 3x = 5$, then $3x - 5$ is a factor

	$$
		\therefore 3x^2 - 17x + 20 = (3x -5)(x - 4)
	$$
\end{example}

\section{Simplifying Algebraic Fractions}

\begin{definition}[Algebraic Fraction]
	An algebraic fraction is an expression of the form:

	$$
		\frac{f(x)}{g(x)}
	$$

	Where $f(x), g(x)$ are expressions in $x$.
\end{definition}
\\
Algebraic fractions are simplified in much the same way as numerical fractions,
and they are added, subtracted, multiplied and divided in the same way also.
The simplification of fractions can be described with a number of rules:

\begin{enumerate}
	\item Fractions can be added once they have a common denominator.
	\item A fraction can be simplified iff the numerator and denominator have a common factor (this is sometimes referred to as cancelling).
	\item If the denominator or numerator contains a fraction, they should be reduced before proceeding.
	\item Division of fractions is equivalent to multiplying the numerator by the inverse of the denominator.
\end{enumerate}

\subsection{Adding Algebraic Fractions}

Fractions can be added once they have a common denominator, thus

\begin{theorem}[Addition of fractions]
	$$
		\frac{a}{b} + \frac{c}{d} = \frac{ad + bc}{bd}
	$$
\end{theorem}

\begin{proof}
	\begin{align}
		\frac{a}{b} + \frac{c}{d}
		  & = \frac{a}{b} \cdot \frac{d}{d} + \frac{c}{d} \cdot \frac{b}{b} \label{multiplicive-identities} \\
		  & = \frac{ad}{bd} + \frac{bc}{bd}                                                                 \\
		  & = \frac{ad + bc}{bd}
	\end{align}
\end{proof}


It should be noted that while the denominator in the above expressions is $bd$,
$\text{LCM}(b,d)$ can be used instead, replacing $b/b$ and $d/d$ in \ref{multiplicive-identities}
with their correct multiplicative identities.

\begin{example}[Simplify]
	$$
		\frac{1}{x+2} + \frac{1}{x+3}
	$$\\
	\textbf{Answer:}

	\begin{align*}
		  & \frac{1}{x+2} + \frac{1}{x+3}    \\
		= & \frac{(x+2) + (x+3)}{(x+2)(x+3)} \\
		= & \frac{2x+5}{(x+2)(x+3)}
	\end{align*}
\end{example}

It should be noted that in most cases it is not necessary to expand
fully the expression in the denominator of the fraction, and usually can be
left in factored form.

\subsection{Reducing Algebraic Fractions}

\begin{theorem}[Reduction of fractions]
	A fraction can be simplified iff the numerator and denominator have a common factor.
	This is sometimes referred to as cancelling.
\end{theorem}
\\
You may need to factor the numerator and/or the denominator to
find a common factor.

\begin{example}[Simplify]
	$$
		\frac{2x^2 - 5x - 3}{4x^2 - 1}
	$$
	\\
	\textbf{Answer:}

	\begin{align*}
		\frac{2x^2 - 5x - 3}{4x^2 - 1}
		  & = \frac{(2x + 1)(x-3)}{(2x + 1)(2x - 1)}                   \\
		  & = \frac{\cancel{(2x + 1)}(x-3)}{\cancel{(2x + 1)}(2x - 1)} \\
		  & = \frac{x - 3}{2x - 1}
	\end{align*}
\end{example}

\end{document}