\documentclass{article}
  \usepackage[landscape]{geometry}
  \usepackage{url}
  \usepackage{multicol}
  \usepackage{amsmath}
  \usepackage{esint}
  \usepackage{amsfonts}
  \usepackage{tikz}
  \usetikzlibrary{decorations.pathmorphing}
  \usepackage{amsmath,amssymb}
  
  \usepackage{colortbl}
  \usepackage{xcolor}
  \usepackage{mathtools}
  \usepackage{polynom}
  \usepackage{amsmath,amssymb}
  \usepackage{enumitem}
  \makeatletter
  
  \newcommand*\bigcdot{\mathpalette\bigcdot@{.5}}
  \newcommand*\bigcdot@[2]{\mathbin{\vcenter{\hbox{\scalebox{#2}{$\m@th#1\bullet$}}}}}
  \makeatother
  
  \title{130 Cheat Sheet}
  \usepackage[english]{babel}
  \usepackage[utf8]{inputenc}
  
  \advance\topmargin-.8in
  \advance\textheight3in
  \advance\textwidth3in
  \advance\oddsidemargin-1.5in
  \advance\evensidemargin-1.5in
  \parindent0pt
  \parskip2pt
  \newcommand{\hr}{\centerline{\rule{3.5in}{1pt}}}
  %\colorbox[HTML]{e4e4e4}{\makebox[\textwidth-2\fboxsep][l]{texto}
  \begin{document}
  
  \begin{center}{\huge{\textbf{Algebra 1 Cheat Sheet}}}\\
  \end{center}
  \begin{multicols*}{3}
  
  \tikzstyle{mybox} = [draw=black, fill=white, very thick,
      rectangle, rounded corners, inner sep=10pt, inner ysep=10pt]
  \tikzstyle{fancytitle} =[fill=black, text=white, font=\bfseries]
  
  %------------ Polynomials ---------------
  \begin{tikzpicture}
  \node [mybox] (box){%
      \begin{minipage}{0.3\textwidth}
      An expression of the form:
      
      $$
        a_x x^n + a_{n-1} x^{n - 1} + \dots + a^2 x^2 + a_1 x + a_0
      $$
      
      Where $a_i \in \mathbb{R}$; $i = 0, 1, 2, \dots, n$; $n \in \mathbb{N}$.
      \end{minipage}
  };
  %------------ Polynomials Header ---------------------
  \node[fancytitle, right=10pt] at (box.north west) {Polynomial};
  \end{tikzpicture}
  
  %------------ Types Of Polynomials ---------------
  \begin{tikzpicture}
  \node [mybox] (box){%
      \begin{minipage}{0.3\textwidth}
          \small{
          \begin{tabular}{lp{4cm} l}
          Linear \emph{(degree 1)} & $ax + b$ \\ \hline
          Quadratic \emph{(degree 2)} & $ax^2 + bx + c$ \\ \hline
          Cubic \emph{(degree 3)} & $ax^3 + bx^2 + cx + d$     \end{tabular}}
      \end{minipage}
  };
  %------------ Types Of Polynomials Header ---------------------
  \node[fancytitle, right=10pt] at (box.north west) {Types Of Polynomials};
  \end{tikzpicture}
  
  %------------ Factoring ---------------
  \begin{tikzpicture}
  \node [mybox] (box){%
      \begin{minipage}{0.3\textwidth}
      \small{
          \begin{tabular}{lp{3.85cm} l}
          Difference Of Two Squares & $a^2 - b^2 = (a + b)(a - b)$ \\ \hline
          Difference Of Two Cubes & $a^3 - b^3 = (a- b)(a^2 + ab + b^2)$ \\ \hline
          Sum Of Two Cubes & $a^3 + b^3 = (a + b)(a^2 -ab + b^2)$     \end{tabular}}
      \end{minipage}
  };
  %------------ Factoring Header ---------------------
  \node[fancytitle, right=10pt] at (box.north west) {Factoring};
  \end{tikzpicture}
  
  %------------ Polynomial Long Divison ---------------
  \begin{tikzpicture}
  \node [mybox] (box){%
      \begin{minipage}{0.3\textwidth}
        $(x^3 + x^2 - 2x) \div (x - 1)$\\
        Solution:
        \begin{center}
        \polylongdiv{x^3 + x^2 - 2x}{x - 1}
      \end{center}
      \end{minipage}
  };
  %------------ Polynomial Long Division Header ---------------------
  \node[fancytitle, right=10pt] at (box.north west) {Polynomial Long Division};
  \end{tikzpicture}
  %------------ Algebraic Fractions ---------------------
  \begin{tikzpicture}
  \node [mybox] (box){%
      \begin{minipage}{0.3\textwidth}
        \begin{align*}
          a(\frac{b}{c}) &= \frac{ab}{c}   &  
          \frac{(\frac{a}{b})}{c} &= \frac{a}{bc}  \\
          \frac{a}{(\frac{b}{c})} &= \frac{ac}{b}   &  
          \frac{(\frac{a}{b})}{(\frac{c}{d})} &= \frac{ad}{bc}\\
          \frac{a}{b} + \frac{c}{d} &= \frac{ad + bc}{bd}   &  
          \frac{a+b}{c} &= \frac{a}{c} + \frac{b}{c} \\
          \frac{ab + ac}{a} &= b + c, a \neq 0
        \end{align*}
    \end{minipage}
  };
  %------------ Algebraic Fractions Header ---------------------
  \node[fancytitle, right=10pt] at (box.north west) {Algebraic Fractions};
  \end{tikzpicture}
  
  %------------ Binomial Coefficient ---------------
  \begin{tikzpicture}
  \node [mybox] (box){%
      \begin{minipage}{0.3\textwidth}
      The binomial coefficient represents the number of ways you can choose $r$ objects from a group of $n$ objects.
      
      \begin{align*}
        {n \choose r } &= \frac{n!}{r!(n - r)!} & {n \choose r} &= {n \choose n - r}
      \end{align*}
      
      $$
      {n\choose r} =  \frac{n(n-1)(n-2)\dots(n-(k - 1))}{k(k-1)(k-2) \dots 1}
      $$
      \end{minipage}
  };
  %------------ Binomial Coefficient Header ---------------------
  \node[fancytitle, right=10pt] at (box.north west) {Binomial Coefficient};
  \end{tikzpicture}
  
  %------------ Binomial Theorem ---------------
  \begin{tikzpicture}
  \node [mybox] (box){%
      \begin{minipage}{0.3\textwidth}
        For any positive integer $n$,
      \begin{align*}
        (a + b)^n &= \sum_{m = 0}^{n} {n\choose m} a^{n - m} b^{n}\\
        &= {n\choose 0} a^n + {n\choose 1}a^{n - 1}b^{1} + \dots + {n\choose n}b^n
      \end{align*}
      \end{minipage}
  };
  %------------ Binomial Theorem Header ---------------------
  \node[fancytitle, right=10pt] at (box.north west) {Binomial Theorem};
  \end{tikzpicture}
  
  %------------ Algebraic Identities ---------------
  \begin{tikzpicture}
  \node [mybox] (box){%
      \begin{minipage}{0.3\textwidth}
      In an \textbf{identity}, all coefficients of like powers are equal.\\
      An identity must be true for \emph{all values} of the independent variables\\\\
      \textbf{Example}: $3x + 7 = ax + b$ implies $a = 3$ and $b = 7$.
  
      \end{minipage}
  };
  %------------ Algebraic Identities Header ---------------------
  \node[fancytitle, right=10pt] at (box.north west) {Algebraic Identities};
  \end{tikzpicture}
  
  %------------ Manipulating Formulae ---------------
  \begin{tikzpicture}
  \node [mybox] (box){%
      \begin{minipage}{0.3\textwidth}
      \textbf{What you do to one side of an equation, you must do to the other.}\\
     
     \small{
          \begin{tabular}{lp{3.85cm} l}
          In terms of $x$  & Write with $x$ as the independent variable  \\ \hline
          The subject of & Make it on it's own on one side of the equation \\ \hline
          As a function of & same as 'in terms of'\end{tabular}}

      \end{minipage}
  };
  %------------ Manipulating Formulae Header ---------------------
  \node[fancytitle, right=10pt] at (box.north west) {Manipulating Formulae};
  \end{tikzpicture}\\
  
   %------------ Solving Linear Equations  ---------------
  \begin{tikzpicture}
  \node [mybox] (box){%
      \begin{minipage}{0.3\textwidth}
      Linear equations can be solved by making x with subject of the equation.\\
 	  Example: Solve $4x + 2 = 14$
 	  \begin{align*}
 	   4x + 2 &= 14\\
 	    \implies 4x &= 14 - 2 = 12\\
 	    \implies x &= \frac{12}{4} = 3 
      \end{align*}
      \end{minipage}
  };
  %------------ Solving Linear Equations Header ---------------------
  \node[fancytitle, right=10pt] at (box.north west) {Solving Linear Equations};
  \end{tikzpicture}
  
  %----------- Solving Linear Systems Of Equations  ---------------
  \begin{tikzpicture}
  \node [mybox] (box){%
      \begin{minipage}{0.3\textwidth}
      1. Reduce the three equations to two by eliminating one of the unknowns.\\
      2. Choose an unknown to isolate.\\
      3. Eliminate this unknown from all three equations, taking them two at a time.\\
      4. Solve these two equations.\\
      5. Use the solutions as values in the original equations\\
      6. Check your solution!
      \end{minipage}
  };
  %------------ Solving Linear Systems Of Equations Header ---------------------
  \node[fancytitle, right=10pt] at (box.north west) {Systems of Equations With Three Variables};
  \end{tikzpicture}
  
  \end{multicols*}
  \end{document}