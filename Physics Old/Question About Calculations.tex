\documentclass[border=1cm]{standalone}

  \usepackage{tikz}
  \usetikzlibrary{optics}
  \usetikzlibrary{calc}
  \usetikzlibrary{shapes}
  
  \begin{document}
  
  \begin{tikzpicture}[use optics]
    % mirror
    \node[
      spherical mirror, 
        name=M,  
        object height=10cm,
        spherical mirror angle=from_radius(6cm),
        draw mirror focus={label=below:$f$}, 
        draw mirror center={red, label=below:$c$}
    ] (M) at (0,0) {};
    
    % Mirror intersection
    \coordinate (P1) at (M.45); %(M.arc center |- M.22);
  
    % baseline
    \draw (M.arc center) -- ($ (M.mirror center) - (3,0) $);
    
    % object
    \coordinate (Obj_Start) at ($ (M.mirror center)$);
    \coordinate (Obj_End) at (Obj_Start |- P1);
    \draw [->] (Obj_Start) -- (Obj_End) node[midway,left] {Object};
    
    % calculating the location of the image
    \coordinate (P2) at (M.arc center);
    \path let \p1 = (Obj_End) in coordinate (Obj_Inv) at (\x1,-\y1);
    \coordinate (Image_End) at (intersection of P1--M.focus and P2--Obj_Inv);
    
    
    % image
    \coordinate (Image_Start) at (Image_End |- Obj_Start);
    \draw [->] (Image_Start) -- (Image_End) node[midway,left] {Image};
    
    \begin{scope}[red]
      % ray from the top of the object to the mirror, 
      % parallel with the baseline
        \draw[->-] (Obj_End) -- (P1);
        
        % ray through the focus
        \draw[->, shorten >=-2cm] (P1) -- (Image_End);  		
    \end{scope}
    
    \begin{scope}[violet]
      % ray to the mirror pole (Incident Ray)
        \draw[->-] (Obj_End) -- (P2);
      
      % ray to the image (Reflected Ray)
        \draw[->, shorten >=-2cm] (P2) -- (Image_End);
    \end{scope}
    
    
  \end{tikzpicture}
  
  
  
  
  
  %\begin{tikzpicture}[use optics]
  %	% mirror
  %	\node[
  %		spherical mirror, 
  %	  	name=M,  
  %	  	object height=4cm,
  %	  	spherical mirror angle=from_radius(5cm),
  %	  	draw mirror focus={label=below:$f$}, 
  %	  	draw mirror center={red, label=below:$c$}
  %	] (M) at (0,0) {};
  %
  %	% baseline
  %	\draw (M.arc center) -- ($ (M.mirror center) - (1.5,0) $);
  %\end{tikzpicture}
  \end{document}
  
  %\begin{tikzpicture}[use optics] % mirror with hole \begin{scope}
  %  \clip (-0.75cm,-2.2cm) rectangle (1cm,0-0.33cm) (-0.75cm,2.2cm) rectangle (1cm,0+0.33cm);
  %\node[spherical mirror, object height=4cm, spherical mirror angle=50] (M1) at (0cm,0) {}; \end{scope}
  %% small mirror
  %\node[convex mirror, spherical mirror orientation=rtl,
  %object height=1cm, spherical mirror angle=90] (M2) at (-4cm,0) {};
  %% convergence point
  %\coordinate (F) at (1cm,0);
  %% red ray
  %\begin{scope}[red]
  %  \draw[-<-] (M1.22) coordinate (P1) -- +(-5cm,0);
  %  \draw[->-] (P1) -- (M2.30) coordinate (Q1);
  %  \draw[->-] (Q1) -- ($(Q1)!1.25!(F)$) coordinate (R1);
  %\end{scope}
  %% blue ray
  %\begin{scope}[blue]
  %  \draw[-<<-] (M1.-22) coordinate (P2) -- +(-5cm,0);
  %  \draw[->>-] (P2) -- (M2.-30) coordinate (Q2);
  %  \draw[->>-] (Q2) -- ($(Q2)!1.25!(F)$) coordinate (R2);
  %\end{scope}
  %% violet ray
  %\begin{scope}[violet]
  %  \draw[-<-] (M1.22) coordinate (P3) -- +(175:5cm);
  %  \draw (P3) -- (M2.22) coordinate (Q3);
  %  \draw (Q3) -- ($(Q3)!1.25!($(F)+(0,-0.15cm)$)$) coordinate (R3);
  %\end{scope}
  %% sensor
  %\node[generic sensor, anchor=aperture west] at ($(R1)!0.5!(R2)$) {}; \end{tikzpicture}
  