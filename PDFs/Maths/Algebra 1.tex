\documentclass[a4paper,titlepage,onecolumn,superscriptaddress,12pt,unpublished]{quantumarticle} % For a longer style, uncomment this
  \pdfoutput=1
  \usepackage[utf8]{inputenc}
  \usepackage[english]{babel}
  \usepackage[T1]{fontenc}
  \usepackage{amsmath}
  \usepackage{amsthm}
  \usepackage{hyperref}
  \usepackage{amssymb}
  \usepackage{cancel}
  \usepackage{enumerate}
  \usepackage{tikz}
  \usepackage{float}
  \usepackage{longtable}
  \usepackage{booktabs}
  \usepackage{chngcntr}
  \usepackage[tikz]{mdframed}

  % More ergonomic figure placement for markdown
  \floatplacement{figure}{H}
  \floatplacement{table}{H}

  
    
    \usepackage{polynom}
  
  % \newtheorem{theorem}{Theorem}
  % \newtheorem{definition}{Definition}[section]

  % Environment Colours
  \definecolor{definitiontitle}{RGB}{61,170,61}
  \definecolor{definitiontitleback}{RGB}{216,233,213}

  \definecolor{theoremtitle}{HTML}{7c98b3}
  \definecolor{theoremtitleback}{HTML}{bcdeff}

  \definecolor{exampletitle}{HTML}{ad5a5a}
  \definecolor{exampletitleback}{HTML}{d19c9c}
  
  
  \newcounter{mdexample}
  \counterwithin{mdexample}{section}
  \newcounter{mdthm}
  \counterwithin{mdthm}{section}
  % \newcounter{mddefinition}
  % \counterwithin{mddefinition}{section}
  
  % Example Environment 
  \newenvironment{example}[1]{
    \stepcounter{mdexample}
    \begin{mdframed}[
      frametitle=#1,
      frametitlefont=\itshape,
      topline=false,
      bottomline=false,
      rightline=false,
      linecolor=exampletitleback,
      linewidth=2pt,
      singleextra={
        \node[
          overlay,
          outer sep=0pt,
          anchor=north east,
          text width=2.5cm,
          minimum height=4ex,
          fill=exampletitleback,
          font=\color{exampletitle}\sffamily\scshape
        ] at (O|-P) {example~\themdexample};
        },
      firstextra={
        \node[
          overlay,
          outer sep=0pt,
          anchor=north east,
          text width=2.5cm,
          minimum height=4ex,
          fill=exampletitleback,
          font=\color{exampletitle}\sffamily\scshape
        ] at (O|-P) {example~\themdexample};
        }
      ]
    }
    {\end{mdframed}}

  % Theorem Environment
  \newenvironment{theorem}[1]
  {\stepcounter{mdthm}\begin{mdframed}[
    font=\itshape,
    frametitle={#1.},
    frametitlefont=\bfseries,
    topline=false,
    bottomline=false,
    rightline=false,
    linecolor=theoremtitleback,
    linewidth=2pt,
    singleextra={
      \node[
        overlay,
        outer sep=0pt,
        anchor=north east,
        text width=2.5cm,
        minimum height=4ex,
        fill=theoremtitleback,
        font=\color{theoremtitle}\sffamily\scshape
      ] at (O|-P) {theorem~\themdthm};
      },
    firstextra={
      \node[
        overlay,
        outer sep=0pt,
        anchor=north east,
        text width=2.5cm,
        minimum height=4ex,
        fill=theoremtitleback,
        font=\color{theoremtitle}\sffamily\scshape
      ] at (O|-P) {theorem~\themdthm};
      }
    ]
  }
  {\end{mdframed}}

  % Definition Environment
  \newenvironment{definition}[1]
  {
    %\stepcounter{mdthm}
    \begin{mdframed}[
    frametitle={#1.},
    frametitlefont=\bfseries,
    font=\itshape,
    topline=false,
    bottomline=false,
    rightline=false,
    linecolor=definitiontitleback,
    linewidth=2pt,
    singleextra={
      \node[
        overlay,
        outer sep=0pt,
        anchor=north east,
        text width=2.5cm,
        minimum height=4ex,
        fill=definitiontitleback,
        font=\color{definitiontitle}\sffamily\scshape
      ] at (O|-P) {definition};
      },
    firstextra={
      \node[
        overlay,
        outer sep=0pt,
        anchor=north east,
        text width=2.5cm,
        minimum height=4ex,
        fill=definitiontitleback,
        font=\color{definitiontitle}\sffamily\scshape
      ] at (O|-P) {definition};
      }
    ]
  }
  {\end{mdframed}}
  
  \newcommand\Solution{\par\textbf{\textsf{Solution}}\par\medskip}
  \renewcommand\qedsymbol{QED}

  \setlength{\parindent}{0in}
  \providecommand{\tightlist}{} % This is a command used by pandoc, you can define if needed.
  
\begin{document}


\title{Algebra 1 - Leaving Cert Higher Level Maths}

\date{\today}

\author{Adam Kelly}

\maketitle


\begin{abstract}
  The following is notes on the Algebra 1 section of the leaving cert
  higher level maths course. These notes are not endorsed by any teachers,
  and I have modified them and added my own details to these after the
  classes. All errors are almost surely mine.
  
  You should note that while there may be proofs provided for some
  statements and theorems, they do not need to be learned for the purpose
  of examination. These non-essential proofs are just to increase the
  understanding of the statements being provided.
\end{abstract}

\section{Polynomial Expressions}\label{polynomial-expressions}

\begin{definition}{Polynomial}
    An expression of the form:

    $$
        a_x x^n + a_{n-1} x^{n - 1} + \dots + a^2 x^2 + a_1 x + a_0
    $$
    
    Where $a_i \in \mathbb{R}$; $i = 0, 1, 2, \dots, n$; $n \in \mathbb{N}$.
\end{definition}

\begin{definition}{Degree}
    The highest power of the monomials in a polynomial, with non-zero coefficients.
\end{definition}

There are a few specific types of polynomials which you need to know.

\begin{itemize}
\tightlist
\item
  A \textbf{linear} polynomial is of degree 1 (\(ax + b\))
\item
  A \textbf{quadratic} polynomial is of degree 2 (\(ax^2 + bx + c\))
\item
  A \textbf{cubic} polynomial is of degree 3 (\(ax^3 + bx^2 + cx + d\))
\end{itemize}

There are also names for polynomials based on how many terms they have.

\begin{itemize}
\tightlist
\item
  A \textbf{monomial} is a polynomial containing a single term.
\item
  A \textbf{binomial} is a polynomial containing two terms.
\item
  A \textbf{trinomial} is a polynomial containing three terms.
\end{itemize}

\begin{definition}{Coefficent}
    A quantity that multiplies the variable in an algebraic expression (for example $4$ in $4x^3$).
\end{definition}

\begin{definition}{Constant Term}
    A term in an algebraic expression that does not contain any modifiable variables.
\end{definition}

\subsection{Addition and Subtraction of Polynomial
Expressions}\label{addition-and-subtraction-of-polynomial-expressions}

Adding and subtracting polynomials involves the combination of like
terms by adding their coefficents.

\begin{definition}{Like terms}
    Monomials that contain the same variable raised to the same power.
\end{definition}

\begin{example}{Simplify the following expression:
    $$
        (6x^{2} - 7x + 4) + (7x^{2} - 9x + 8)
    $$
}
    \Solution

    \begin{align*}
         & (6x^{2} - 7x + 4) + (7x^{2} - 9x + 8)  \\
         & = (6x^2 + 7x^2) + (-7x - 9x) + (4 + 8) \\
         & = 13x^2 - 16x^2 + 12
    \end{align*}
\end{example}

\subsection{Multiplying Polynomial
Expressions}\label{multiplying-polynomial-expressions}

Polynomials are multiplied using the distributive property of addition:

\[
    a(b + c) = ab + ac
\]

\begin{example}{Simplify the following expression: 
    $$(x + 4)(2x + 5)$$
}
    \Solution

    \begin{align*}
         & (x + 4)(2x + 5)       \\
         & = 2x^2 + 8x + 5x + 20 \\
         & = 2x^2 + 13x + 20
    \end{align*}
\end{example}

\subsection{Dividing Polynomial
Expressions}\label{dividing-polynomial-expressions}

There are a few cases for the division of polynomials. Some of these are
detailed below, in the form of worked examples.

Remember, the following equalities \textbf{do not hold} when the
quotient's denominator is zero. For example, if you have the expression
\((2x + 2) \div (x + 1)\), and you simplify it to \(2\), you must
remember that this expression is \textbf{not equal} to two if
\(x = -1\), as that would be division by zero and thus undefined.

\subsubsection{Denominator is a Factor of Each
Term}\label{denominator-is-a-factor-of-each-term}

If the denominator is a factor of each term of the numerator, the
quotient can be simplified by dividing each term of the numerator by the
denominator.

\begin{example}{Simplify the following expression: 
    $$
        \frac{-16x^4 + 8x^2 + 2x}{2x}
    $$}
    \Solution
    \begin{align*}
         & \frac{-16x^4 + 8x^2 + 2x}{2x}                         \\
         & = \frac{-16x^4}{2x} + \frac{8x^2}{2x} + \frac{2x}{2x} \\
         & = -7x^3 + 4x + 1
    \end{align*}
\end{example}

\subsubsection{Denominator is a Factor of the
Numerator}\label{denominator-is-a-factor-of-the-numerator}

If the denominator is a factor of the numerator, the quotient can be
simplified by factoring the numerator and then canceling out the factor.

\begin{example}{Simplify the following expression:
    $$
        \frac{15x^2 + 22x + 8}{5x + 4}
    $$}
    \Solution
    \begin{align*}
         & \frac{15x^2 + 22x + 8}{5x + 4}                      \\
         & = \frac{(3x + 2)\cancel{(5x + 4)}}{\cancel{5x + 4}} \\
         & = 3x + 2
    \end{align*}
\end{example}

\subsubsection{Polynomial Long Division}\label{polynomial-long-division}

If the other two methods don't work, you can use polynomial long
division. This is a generalized version of the arithmetic technique
called long division.

\begin{example}{
    \label{long-div-example}
    Simplify the following expression:
    $(x^3 + x^2 - 2x) \div (x - 1)$
}
    \Solution

    \begin{center}
        \polylongdiv{x^3 + x^2 - 2x}{x - 1}
    \end{center}
    
    $$
        \therefore \frac{x^3 + x^2 - 2x}{x - 1} = x^2 + 2x
    $$
\end{example}

You should note that it is possible that it doesn't divide in evenly, in
which case the division will leave a \emph{remainder}. This can just be
added at the end. Here is an example of a remainder.

\begin{example}{Simplify the following expression:
    $$
        \frac{x^3 + x^2 - 1}{x - 1}
    $$}
    
    \Solution
    \begin{center}
        \polylongdiv{x^3+x^2-1}{x-1}
    \end{center}
    
    $$
        \therefore \frac{x^3 + x^2 - 1}{x - 1} = x^2 + 2x + 2 + \frac{1}{x - 1}
    $$

\end{example}

\subsection{Synthetic Division}\label{synthetic-division}

If the denominator of the expression is \(x - a\), where
\(a \in \mathbb{R}\), then you can use a division technique called
synthetic division.

This technique is generally used in the process of finding the roots of
a polynomial.

\begin{example}{
    Complete example \ref{long-div-example} using synthetic division.
}
    \Solution
    \begin{center}
        \polyhornerscheme[x=1]{x^3 + x^2 - 2x}
    \end{center}
    
    $$
        \therefore \frac{x^3 + x^2 - 2x}{x - 1} = x^2 + 2x
    $$

\end{example}

\section{\texorpdfstring{Factorising Polynomial Expressions
\label{factoring}}{Factorising Polynomial Expressions }}\label{factorising-polynomial-expressions}

\begin{definition}{Factor}
    An algebraic factor is an expression that divides a polynomial leaving no remainder.
\end{definition}

Factorising is a very important part of the LC maths course, not to
mention algebra in general. It is very important that you get good at
this skill and become familiar with the common techniques and patterns
that come up. A few factoring techniques are shown below, most in the
form of worked examples.

\subsection{Highest Common Factor}\label{highest-common-factor}

This method can be done by looking for the highest common factor of each
of the expressions and then working backwards from there.

\begin{example}{}
    \begin{align*}
        2x^2 + 6x          & = 2x(x + 3)       \\
        15axy + 2xyb + 4xy & = 2xy(7a + b + 2)
    \end{align*}
\end{example}

\subsection{Grouping}\label{grouping}

An expression can sometimes be factored by breaking the expression up
into groups and factoring the individual parts.

\begin{example}{Factor the expression $6x^2y + 3xy^2 -12x - 6y$}
    \Solution
    \begin{align*}
         & 6x^2y + 3xy^2 -12x - 6y   \\
         & = 3xy(2x + y) - 6(2x + y) \\
         & = (2x + y)(2xy - 5)
    \end{align*}
\end{example}

\subsection{Difference of Two Squares}\label{difference-of-two-squares}

\begin{theorem}{Difference of Two Squares}
    An expression of the form $a^2 - b^2$ can be factored into $(a + b)(a - b)$.
\end{theorem}

\begin{proof}
    \begin{align*}
        (a + b)(a - b)
         & = a(a - b) + b(a - b) \\
         & = a^2 - ab + ab -b^2  \\
         & = a^2 - b^2
    \end{align*}
\end{proof}

\begin{example}{}
    \begin{align*}
        x^2 - 36   & = (x + 6)(x - 6)                                    \\
        4x^2 - 81  & = (2x + 9)(2x - 9)                                  \\
        9x^3 - 81x & = 9x(x^2 - 9) = 9x(x + 3)(x - 3)                    \\
        16x^4 - 1  & = (4x^2 + 1)(4x^2 - 1) = (4x^2 + 1)(2x + 1)(2x - 1) \\
    \end{align*}
\end{example}

\subsection{Difference of Two Cubes}\label{difference-of-two-cubes}

\begin{theorem}{Difference of Two Cubes}
    An expression of the form $a^3 - b^3$ can be factored into $(a - b)(a^2 + ab + b^2)$
\end{theorem}

\begin{proof}
    \begin{align*}
        (a-b)(a^2 + ab +b^2 )
         & = a(a^2 + ab +b^2 ) - b(a^2 + ab +b^2 )                        \\
         & = (a^3 + a^2 b + ab^2) - (a^2b + ab^2 + b^3)                   \\
         & = (a^3 + \cancel{a^2 b + ab^2}) - (\cancel{a^2b + ab^2} + b^3) \\
         & = a^3 - b^3
    \end{align*}
\end{proof}

\begin{example}{}
    \begin{align*}
        x^3 - 8        & = (x - 2)(x^2 + 2x + 4)          \\
        125x^3 - 27y^3 & = (5x - 3y)(25x^2 + 15xy + 9y^2)
    \end{align*}
\end{example}

\subsection{Sum of Two Cubes}\label{sum-of-two-cubes}

\begin{theorem}{Sum Of Two Cubes}
    An expression of the form $a^3 + b^3$ can be factored into $(a + b)(a^2 - ab + b^2)$
\end{theorem}

\begin{proof}
    \begin{align*}
        (a+b)(a^2 - ab +b^2)
         & = a(a^2 - ab +b^2 ) + b(a^2 - ab +b^2 )          \\
         & = a^3 - a^2b + ab^2 + ba^2 - ab^2 + b^3          \\
         & = a^3 - \cancel{a^2b + ab^2 + ba^2 - ab^2} + b^3 \\
         & = a^3 + b^3
    \end{align*}
\end{proof}

\begin{example}{}
    \begin{align*}
        27x^3 + 1      & = (3x + 1)(9x^2 - 3x + 1)         \\
        81x^3 + 192b^3 & = (5x + 6b)(25x^2 - 30xb + 36b^2)
    \end{align*}
\end{example}

\subsection{Factoring With The Quadratic
Formula}\label{factoring-with-the-quadratic-formula}

If the previous methods don't work and you with to factor a quadratic,
you can use ths following method (shown in worked example).

\begin{example}{Factor 
    $3x^2 - 17x + 20$}
    \Solution
    
    First, you must let the expression equal $0$, and then solve for $x$:
    
    \begin{align*}
         & 3x^2 - 17x + 20 = 0                        \\
         & \implies x
        = \frac{17 \pm \sqrt{(-17)^2 - 4(3)(20)}}{2(3)}
        = \frac{17 \pm 7}{6}                          \\
         & \implies x = 4 \text{ or } x = \frac{5}{3}
    \end{align*}
    
    From this you can find that if $x = 4$, then $x- 4$ is a factor.
    If $x = \frac{5}{3} \implies 3x = 5$, then $3x - 5$ is a factor
    
    $$
        \therefore 3x^2 - 17x + 20 = (3x -5)(x - 4)
    $$
\end{example}

\section{Simplifying Algebraic
Fractions}\label{simplifying-algebraic-fractions}

\begin{definition}{Algebraic Fraction}
    An algebraic fraction is an expression of the form:

    $$
        \frac{f(x)}{g(x)}
    $$
    
    Where $f(x), g(x)$ are expressions in $x$.
\end{definition}

Algebraic fractions are simplified in much the same way as numerical
fractions, and they are added, subtracted, multiplied and divided in the
same way also. The simplification of fractions can be described with a
number of rules:

\begin{itemize}
\tightlist
\item
  Fractions can be added once they have a common denominator.
\item
  A fraction can be simplified iff the numerator and denominator have a
  common factor (this is sometimes referred to as cancelling).
\item
  If the denominator or numerator contains a fraction, they should be
  reduced before proceeding.
\end{itemize}

It should be noted that in most cases it is not necessary to expand
fully the expression in the denominator of the fraction, and usually can
be left in factored form.

A few properties can be used during this simplification.

\begin{theorem}{Addition Of Fractions}
    $$
        \frac{a}{b} + \frac{c}{d} = \frac{ad + bc}{bd}
    $$
\end{theorem}

\begin{proof}
    \begin{align*}
        \frac{a}{b} + \frac{c}{d}
         & = \frac{a}{b} \cdot \frac{d}{d} + \frac{c}{d} \cdot \frac{b}{b} \label{multiplicive-identities} \\
         & = \frac{ad}{bd} + \frac{bc}{bd}                                                                 \\
         & = \frac{ad + bc}{bd}
    \end{align*}
\end{proof}

\begin{theorem}{Fractions In The Numerator/Denominator}
    $$
        \frac{\frac{a}{b}}{\frac{c}{d}} = \frac{ad}{bc}
    $$
\end{theorem}

\begin{proof}
    \begin{align*}
        \frac{\frac{a}{b}}{\frac{c}{d}} & = \frac{a}{b} \cdot \frac{1}{\frac{c}{d}} \\
                                        & = \frac{a}{b} \cdot (\frac{c}{d})^{-1}    \\
                                        & = \frac{a}{b} \cdot \frac{d}{c}           \\
                                        & = \frac{ad}{bc}
    \end{align*}
\end{proof}

Here are some examples of following these rules to simplify algebraic
fractions.

\begin{example}{Simplify the following expression:
    $$
        \frac{1}{x+2} + \frac{1}{x+3}
    $$}
    \Solution
    
    \begin{align*}
          & \frac{1}{x+2} + \frac{1}{x+3}    \\
        = & \frac{(x+2) + (x+3)}{(x+2)(x+3)} \\
        = & \frac{2x+5}{(x+2)(x+3)}
    \end{align*}
\end{example}

\begin{example}{Simplify the following expression:
    $$
        \frac{2x^2 - 5x - 3}{4x^2 - 1}
    $$
}
    \Solution

    \begin{align*}
        \frac{2x^2 - 5x - 3}{4x^2 - 1}
         & = \frac{(2x + 1)(x-3)}{(2x + 1)(2x - 1)}                   \\
         & = \frac{\cancel{(2x + 1)}(x-3)}{\cancel{(2x + 1)}(2x - 1)} \\
         & = \frac{x - 3}{2x - 1}
    \end{align*}
\end{example}

\section{Binomial Expansion}\label{binomial-expansion}

\subsection{The Binomial Coefficient}\label{the-binomial-coefficient}

To understand the binomial expansion theorem, you must be familiar with
the binomial coefficient. The binomial coefficient is related to
combinations. The coefficient:

\[
n\choose r
\]

Represents the number of ways you can choose \(r\) objects from a group
of \(n\) objects.

This coefficient is calculated as:

\[
{n\choose r} = \frac{n!}{r!(n-r)!}
\]

There is a number of properties that you will need to know.

\begin{theorem}{}
    $$
    {n\choose r} = {n \choose n - r}
    $$
\end{theorem}

\begin{proof}
    This theorem has an intuitive proof (which you should read!), along with a formal proof.
    To understand how this works, imagine you have $n$ balls, and you are asked to choose 
    $r$ of them to remove from the group. You could also think of this as choosing $n-r$ of them
    to keep. Thus, choosing $r$ objects out of $n$ is equivalent to choosing $n-r$ objects out of $n$.\\
    
    Formally, you can show this theorem algebraically:
    \begin{align*}
        {n\choose r} &= \frac{n!}{r!(n-r)!}\\
        \implies {n\choose n-r} &= \frac{n!}{(n-r)!(n-(n-r))!}\\
        &= \frac{n!}{(n-r)!r!}
    \end{align*}
\end{proof}

It is also possible to calculate this coefficient in an alternate way,
which can be useful for calculating this coefficient without a
calculator.

\[
{n\choose r} =  \frac{n(n-1)(n-2)\dots(n-(k - 1))}{k(k-1)(k-2) \dots 1}
\]

\begin{example}{{Without a calculator, find $$ 7 \choose 4$$}}
    \Solution

    \begin{align*}
    {7 \choose 4} &= \frac{7 \cdot 6 \cdot 5 \cdot 4}{4 \cdot 3 \cdot 2 \cdot 1}\\
    &= \frac{840}{24}   \\
    &= 35
    \end{align*}
\end{example}

\begin{example}{{Without a calculator, find $$ 15 \choose 11$$}}
    \Solution

    \begin{align*}
    {15 \choose 11} &= {15 \choose 4} \\
    &= \frac{15 \cdot 14 \cdot 13 \cdot 12}{4 \cdot 3 \cdot 2 \cdot 1}\\
    &= 1365
    \end{align*}
\end{example}

\subsection{The Binomial Theorem}\label{the-binomial-theorem}

\begin{theorem}{The Binomial Theorem}
For any positive integer $n$,
\begin{align*}
(a + b)^n &= \sum_{m = 0}^{n} {n\choose m} a^{n - m} b^{n}\\
&= {n\choose 0} a^n + {n\choose 1}a^{n - 1}b + {n\choose 2} a^{n - 2}b^2 + {n\choose 3} a^{n - 3}b^3 + \dots + {n\choose n}b^n
\end{align*}
\end{theorem}

\begin{proof}
$\underbrace{(a + b)^n = (a + b)(a + b)(a+b) \dots (a + b)}_{n \text{ times}}$.
Repeatedly using the distributive property, we can see that for a term $a^m b^{n-m}$, we must choose $m$ of the $n$ terms to contribute an $a$ to the term, and then each of the other $n - m$ terms of the products must contribute a $b$ term. Thus the coefficient of the $a^m b^{n-m}$ is the number of ways to choose $m$ objects from a set of size $n$, which is $n \choose m$. For all possible values of $0 \leq m \leq n$, we can see that

$$
(a + b)^n = \sum_{m = 0}^{n} {n \choose m} a^m b^{n - m}
$$
\end{proof}

\section{Algebraic Identities}\label{algebraic-identities}

\begin{definition}{Identity}
In an identity, all coefficients of like powers area equal. An identity must be true for all values of the independent variable.
\end{definition}

When you have an equation and have matching coefficients on both sides,
then they are equal

\begin{example}{
Find the values of $a, b$ and $c$ given

$$
ax^2 + bx + c = 3x^2 + 2x + 5
$$

For all values of $x$.
}
\Solution
\begin{align*}
a = 3, b = 2, c = 5
\end{align*}
\end{example}

\begin{example}{
$$
\frac{1}{(x + 1)(x - 1)} = \frac{A}{(x + 1)} + \frac{B}{(x - 1)}
$$

For all values of $x$, find values of $A$ and $B$.
}
\Solution

\begin{align*}
\frac{1}{(x + 1)(x - 1)} &= \frac{A}{(x + 1)} + \frac{B}{(x - 1)} \\
\implies \frac{1}{(x + 1)(x - 1)} &= \frac{A(x - 1) + B(x + 1)}{(x + 1)(x - 1)} \\
\implies 1 &= A(x - 1) + B(x + 1)
\end{align*}

Since this equation is true for all values of $x$, you can pick two values for x and $A$ and $B$ can be found.

\end{example}

% \onecolumn\newpage
% \appendix

% \section{Calculating \LaTeX}

% Uncomment this if needed!
\end{document}
