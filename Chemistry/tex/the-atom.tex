\hypertarget{the-first-atomic-theory}{%
\section{The First Atomic Theory}\label{the-first-atomic-theory}}

Atomic theory was first proposed by John Dalton, where he had the
theory:

\begin{enumerate}
\def\labelenumi{\arabic{enumi}.}
\tightlist
\item
  All matter is made up of very small particles called atoms
\item
  Atoms are indivisible (can't be broken down into smaller parts, which
  is wrong)
\end{enumerate}

\hypertarget{discovery-of-the-electron}{%
\section{Discovery of the Electron}\label{discovery-of-the-electron}}

\hypertarget{crookes-maltese-cross-experiment}{%
\subsection{Crookes `Maltese Cross'
Experiment}\label{crookes-maltese-cross-experiment}}

\begin{itemize}
\tightlist
\item
  In 1875, William Crookes investigated what happens when electric
  current is passed through a glass tube containing air at low pressure
\item
  He used a glass tube, with a cathode, an anode and an object, along
  with a high voltage power supply
\end{itemize}

\begin{figure}
\centering
\includegraphics{https://upload.wikimedia.org/wikipedia/commons/b/bf/Crookes_tube_two_views.jpg}
\caption{Crookes Tube}
\end{figure}

\begin{itemize}
\tightlist
\item
  He found radiation from the cathode
\item
  Shown by putting an object (a maltese cross) in the glass, which shows
  a `shadow'
\item
  Described these as cathode rays (which are rays of electrons)
\item
  The cathode rays cause a glowing
\end{itemize}

\hypertarget{crookes-paddle-wheel-experiment}{%
\subsection{Crookes `Paddle Wheel'
Experiment}\label{crookes-paddle-wheel-experiment}}

To investigate the properties of these cathode rays, crookes set up
another experiment.

\begin{itemize}
\tightlist
\item
  A light paddle wheel was placed in front of the cathode
\item
  When the current was switched on, the paddle wheel moved
\item
  The wheels vanes always moved away
\item
  The canes were being struck by particles, coming from the cathode
\end{itemize}

\begin{figure}
\centering
\includegraphics{https://physicsmax.com/wp-content/uploads/2014/08/1603.jpg}
\caption{Experiment Diagram}
\end{figure}

The following conclusions were drawn

\begin{itemize}
\tightlist
\item
  Cathode rays travel in straight lines
\item
  Cathode rays cause the glasses phosphor coating to glow
\item
  Cathode rays have enough energy to turn a paddle wheel
\end{itemize}

\hypertarget{thomsons-cathode-ray-experiment}{%
\subsection{Thomson's Cathode Ray
Experiment}\label{thomsons-cathode-ray-experiment}}

At the end of the 19th century, J.J. Thomson tried to resolve some
outlying issues with cathode rays. He constructed a cathode ray tube
that allowed the use of two electrodes to show that the cathode rays had
a negative charge.

\begin{figure}
\centering
\includegraphics{https://upload.wikimedia.org/wikipedia/commons/thumb/a/a3/JJ_Thomson_Cathode_Ray_2_explained.svg/500px-JJ_Thomson_Cathode_Ray_2_explained.svg.png}
\caption{Experiment diagram}
\end{figure}

From this Thomson proposed a \textbf{plum pudding} model of the atom

\hypertarget{millikans-oil-drop-experiment}{%
\subsection{Millikan's Oil Drop
Experiment}\label{millikans-oil-drop-experiment}}

After Thomson's experiment, it was not universally accepted that
subatomic particles exist. Millikan wanted to measure the charge of an
electron (which could also give the mass).

\begin{itemize}
\tightlist
\item
  Tiny droplets of oil were sprayed between two charged metal plates.
\item
  X-rays were used to ionise the air between the plates (the molecules
  in the air lost electrons and formed ions)
\item
  When the oil droplets fell through the air they picked up these
  electrons, becoming negatively charged.
\item
  Using a microscope, a particular oil droplet was focused on, and
  millikan observed that the negatively charged droplet was attracted to
  the positive plate.
\item
  The charge of the plates were adjusted until the oil droplet was
  stationary, and the size of the electron's charge
\end{itemize}

\begin{figure}
\centering
\includegraphics{https://lh6.googleusercontent.com/-4EWgjurl8WM/TXRsQZQ49UI/AAAAAAAAAco/bZfkBqaER3g/s1600/big+mod.JPG}
\caption{Diagram}
\end{figure}

\hypertarget{the-plum-pudding-model-of-the-atom}{%
\subsection{The Plum Pudding Model of the
Atom}\label{the-plum-pudding-model-of-the-atom}}

After his experiments, Thomson proposed a simple model of the atom:

\begin{itemize}
\tightlist
\item
  An atom like a sphere of positive
\item
  Electrons are embedded in the sphere at random.
\end{itemize}

\begin{figure}
\centering
\includegraphics{https://d2gne97vdumgn3.cloudfront.net/api/file/G513BNVbRkaYgB63CplV}
\caption{Plum Pudding}
\end{figure}

\hypertarget{discovery-of-the-nucleus}{%
\subsection{Discovery of the Nucleus}\label{discovery-of-the-nucleus}}

Ernest Rutherford did the experiment of bombarding a sheet of gold with
alpha particles.

\begin{definition}{Alpha Particles}
Positively charged particles produced by certain radioactive substances.
They consist of two neutrons and two protons stuck together.
\end{definition}

\begin{figure}
\centering
\includegraphics{https://i.imgur.com/BToJ11t.jpg}
\caption{The experiment carried out by Rutherford to investigate how
alpha particles were scattered by a piece of gold foil.}
\end{figure}

\begin{itemize}
\tightlist
\item
  A thin piece of gold foil was bombarded with alpha particles
\item
  It was expected that the alpha particles would be deflected by small
  amounts.
\item
  A phosphorescent screen was used to detect the alpha particles
\item
  Rutherford found some alpha particles were deflected at large angles,
  and some were deflected along their own paths.
\end{itemize}

The following conclusions were made.

\begin{longtable}[]{@{}ll@{}}
\toprule
\begin{minipage}[b]{0.45\columnwidth}\raggedright
\textbf{Observation}\strut
\end{minipage} & \begin{minipage}[b]{0.49\columnwidth}\raggedright
\textbf{Conclusion}\strut
\end{minipage}\tabularnewline
\midrule
\endhead
\begin{minipage}[t]{0.45\columnwidth}\raggedright
Most alpha particles pass straight through the gold foil.\strut
\end{minipage} & \begin{minipage}[t]{0.49\columnwidth}\raggedright
Most of the atom is empty space.\strut
\end{minipage}\tabularnewline
\begin{minipage}[t]{0.45\columnwidth}\raggedright
Some alpha particles are deflected at large angles.\strut
\end{minipage} & \begin{minipage}[t]{0.49\columnwidth}\raggedright
The alpha particles are repelled when they pass near the small, positive
nucleus.\strut
\end{minipage}\tabularnewline
\begin{minipage}[t]{0.45\columnwidth}\raggedright
A small number of alpha particles are reflected back along their own
paths.\strut
\end{minipage} & \begin{minipage}[t]{0.49\columnwidth}\raggedright
A small number of alpha particles collide head on with the
nucleus.\strut
\end{minipage}\tabularnewline
\bottomrule
\end{longtable}

\hypertarget{discovery-of-the-proton}{%
\subsection{Discovery of The Proton}\label{discovery-of-the-proton}}

Rutherford continued his experiment of bombardment with alpha particles,
but he switched to nitrogen and oxygen (which are lighter elements)

He found that small positive particles were given off - protons.

\hypertarget{discovery-of-the-neutron}{%
\subsection{Discovery of the Neutron}\label{discovery-of-the-neutron}}

James Chadwick bombarded beryllium, and found similar results. He
discovered some type of radiation made up of particles with no charge
coming from the beryllium. These particles had about the same mass as
the proton, and he named them \emph{neutrons}.

\hypertarget{properties-of-the-sub-atomic-particles}{%
\section{Properties of the Sub-Atomic
Particles}\label{properties-of-the-sub-atomic-particles}}

\begin{longtable}[]{@{}llll@{}}
\toprule
& \emph{Relative Charge} & \emph{Relative Mass} &
\emph{Location}\tabularnewline
\midrule
\endhead
\emph{Proton} & +1 & 1 & Nucleus\tabularnewline
\emph{Neutron} & 0 & 1 & Nucleus\tabularnewline
\emph{Electron} & -1 & 1/1838 & Outside Nucleus\tabularnewline
\bottomrule
\end{longtable}
